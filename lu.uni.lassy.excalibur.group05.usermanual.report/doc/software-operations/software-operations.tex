\chapter{Software operations}
\label{chap:soptware_operations}


Explain each allowed software operations (i.e. an atomic unit of treatment, a service, a functionality) including a brief description of the operation, required parameters, optional parameters, default options, required steps to trigger the operation, assumptions upon request of the operation and expected results of executing such operation.
Describe how to recognise that the operation has successfully been executed or
abnormally terminated. The template given below (i.e. section \ref{operation:MyOperation} has to be used).

Group the operations devoted to the needs of specific actors. Common
operations to several actors may be grouped and presented once to avoid redundancy.


\section{MyOperation}
\label{operation:MyOperation}
The system operator creates and adds a new crisis to the system after being
informed by a third party (citizen, organization) and selects a crisis handler for the crisis.

\begin{description}

\item \textbf{Parameters:} Reporter Personal Information, Crisis Information, Crisis Handler
\item \textbf{Precondition:} The system operator is logged in and has received information from a reporter.
\item \textbf{Post-condition:} A new crisis has been added to the system and the new crisis has been assigned to a crisis handler, the Handler has received an automatic notification from the system.
\item \textbf{Output messages:} The selected Crisis Handler will be notified
automatically once the crisis has been created.

\item \textbf{Triggering:}
\begin{enumerate}
\item From within the crisis management window fill out the required entries related to the personal information of the reporter such as name and phone number.
\item Fill out the entries related to the crisis type, impacted area, priority, description, GPS coordinates, address and finally choose a Crisis Handler from the combo box.
\item Click on the “Submit” button in and add the entry to the database.
\end{enumerate}

 
\end{description}

% \begin{description}
% 
% \item \textbf{Parameters:} 
% \item \textbf{Precondition:} 
% \item \textbf{Post-condition:} 
% \item \textbf{Output messages:} 
% 
% \item \textbf{Triggering:}
% \begin{enumerate}
% \item 
% \item 
% \item 
% \end{enumerate}
% 
% \end{description}

 
\subsection{MyExample1}
Examples should illustrate the use of \textbf{complex operations}.

Each example must show how the actor uses the software operation under
description to achieve (at least one of) its expected outcome.

It might be required to include GUI screenshots to illustrate the example.

\section{Activator}

\subsection{SollicitateCrisisHandling}
\label{operation:sollicitatecrisishandling}

Activator informs coordinators about or randomly allocates crises that stood too
longly in a not handled status.

\begin{description}

\item \textbf{Parameters:} none
\item \textbf{Precondition:} The system is started and there exist some
logged coordinators.
\item \textbf{Post-condition:} If there exist crises who
stood in a not handled status more than the maximum allowed time then those
crises are randomly allocated to the existing coordinators. For all other crisis
who stood too longly in a not handled status but not more than the maximum
delay allowed then a reminder message is sent to the administrator and all
coordinator actors of the environment to sollicitate handling of those crisis.
\item \textbf{Output messages:} reminder message is sent to the administrator
and all coordinator actors of the environment to sollicitate handling of those
crisis

\item \textbf{Triggering:}
\begin{enumerate} 
	\item there exist some crisis that are in pending status and for which the
	duration between the current system's state clock information and the last
	reminder is greater than the crisis reminder period duration.
\end{enumerate}

\end{description}

\section{Administrator}

\subsection{AddCoordinator}
\label{operation:addcoordinator}

Administrator adds new coordinator to the system.

\begin{description}

\item \textbf{Parameters:} Coordinator's ID, Coordinator's Login, Coordinator's
Password, Coordinator's FingerPrint
\item \textbf{Precondition:} the system is started, administrator logged
previously and did not log out and new coordinator provided his fingerprint
information
\item \textbf{Post-condition:} the environment has a new instance of coordinator
actor allowing for input/output message communication with the system, new entry
added to the database, and the administrator is informed about the satisfaction
of its request.
\item \textbf{Output messages:} Administrator will be automatically notified
once new coordinator added

\item \textbf{Triggering:}
\begin{enumerate}
\item From within the crisis management Admin window press ``Add a coordinator''
button
\item Fill out the fields related to new coordinator's id, username, password
and fingerprint information file
\item Click on the ``Create'' button
\end{enumerate}

\end{description}

\subsection{DeleteCoordinator}
\label{operation:deletecoordinator}

Administrator deletes existing coordinator from the system.

\begin{description}

\item \textbf{Parameters:} Coordinator's ID
\item \textbf{Precondition:} the system is started, administrator logged
previously and did not log out, there exists one coordinator with the same Id
as one provided by administrator
\item \textbf{Post-condition:} the coordinator having the
required id do not belong anymore to the system and corresponding entry removed
from the database, the administrator is informed about the satisfaction of his
request.
\item \textbf{Output messages:} The Administrator is automatically notified once
the coordinator has been deleted

\item \textbf{Triggering:}
\begin{enumerate}
\item From within the crisis management Admin window press ``Delete a
coordinator'' button
\item Fill out the field related to ID of coordinator which Administrator wants
to delete
\item Click on the ``Delete'' button
\end{enumerate}

\end{description}

\subsection{MonitorOverallQualityInsurence}

Administrator observes relative amounts of each type of mark-answer
(number from 0 to 5) to given QA survey question for all crises

% TODO synchronize with mockup representation
\begin{description}

\item \textbf{Parameters:} 
\item \textbf{Precondition:}
\item \textbf{Post-condition:}
\item \textbf{Output messages:}

\item \textbf{Triggering:}
\begin{enumerate}
\item
\item
\item
\end{enumerate}

\end{description}

\subsection{MonitorQualityInsurenceForCrisis}

Administrator observes relative amounts of each type of mark-answer
(number from 0 to 5) to given QA survey question for specified crisis

% TODO synchronize with mockup representation
\begin{description}

\item \textbf{Parameters:} 
\item \textbf{Precondition:}
\item \textbf{Post-condition:}
\item \textbf{Output messages:}

\item \textbf{Triggering:}
\begin{enumerate}
\item
\item
\item
\end{enumerate}

\end{description}

\section{Authenticated}

\subsection{Login}

Actor requests authorization to get access to secured system operations.

\begin{description}

\item \textbf{Parameters:} Actor's Login, Actor's Password
\item \textbf{Precondition:} The system is started and actor is not already
logged in
\item \textbf{Post-condition:} If the authentication information is correct then
the actor is known to be logged in.
\item \textbf{Output messages:} If the authentication information is correct,
the message ``You are logged! Welcome \ldots'' is shown to actor. Otherwise -
the message ``Wrong identification information! Please try again \ldots'' is
shown.

\item \textbf{Triggering:}
\begin{enumerate}
\item From within the crisis management Authentication window (whether Admin or
Coordinator window) fill out the fields related to actor's login and password
\item Press ``logon'' button and try to authenticate
\item If login/password pair is correct, open special mobile app to perform
fingerprint scanning, scan finger with the special finger scanning device and
send fingerprint data to the system. If provided fingerprint info is correct,
get welcome message from the system.
\item Otherwise be notified that actor gave incorrect data and all
the administrator actors existing in the environement are notified of an
intrusion temptative.
\end{enumerate}

\end{description}

\subsection{Logout}
Actor logouts to end the secured access to specific system operations.

\begin{description}

\item \textbf{Parameters:} none
\item \textbf{Precondition:} the system is started and the actor is currently
logged in.
\item \textbf{Post-condition:} the actor is known to be logged out
\item \textbf{Output messages:} a logout confirmation message is sent to the
actor

\item \textbf{Triggering:}
\begin{enumerate}
\item From within the crisis management Administrator/Coordinator window press
``Logoff'' button
\end{enumerate}

\end{description}

\section{Communication Company}

\subsection{Alert}

Communication Company delcares and sends structured information (alert) about a
crisis obtained from sms message of some human having a phone able to connect to the
actor.

\begin{description}

\item \textbf{Parameters:} 
\begin{itemize}
  \item the kind of human informing of an lert (either victim, witness or
  anonym)
  \item date of alert
  \item time of alert
  \item the phone number of the human sending the alert SMS message
  \item the GPS position of the phone at the date and time the message was sent
  \item a free text message sent by the human providing information on the alert
  that he wants to declare
\end{itemize}
\item \textbf{Precondition:} the system is supposed to be created and
initialized; the date and time the alert is declared is supposed to be in the
past with respect to the current time known by the system.
\item \textbf{Post-condition:} 
\begin{itemize}
	\item new alert with status ``pending'', instant
information (GPS location and comment) based on date and time provided
(position and comment) is added to the system 
	\item If there exist no already registered alert near to the alert currently
	declared then a new crisis is added to the system and initialized with: type
	considered as small, its status being pending, its declared time being the
	same than the alert and a default comment indicating that a report will come
	later on. Otherwise - the crisis to which the new alert must be related to is
	the one related to any alert nearby.
	\item If there is no human instance having same phone number and same kind in
	the then a new one is added with given phone number and kind and is associated
	to the communication company actor used to declare the alert.
	\item Specified human is related to the new alert thus indicating he has
	signed the alert.
\end{itemize}
 
\item \textbf{Output messages:} If provided information is valid, the message
``Your alert has been registered. We will handle it and keep you informed'' is
shown to the Communication Company once the alert is saved in the system.
Otherwise - ``Incorrect data'' message is shown.

\item \textbf{Triggering:}
\begin{enumerate}
\item From within the crisis management Communication Company window fill out
the required entries related to the information of the alert such type of
person who sent SMS message with info about crisis, date of alert, time of
crisis, phone number of this person, latitude and longtitude of person's phone
location at time the message was sent and comment on alert.
\item Click on ``Send alert'' button and send alert's information to the system.
\end{enumerate}

\end{description}

\subsection{QualityAssuranceQuestionAnswer}

Communication Company declares and sends structured information about customer's
(victim, witness or anonym) answer to QA survey questions sent to him previously
after some crisis, to which the person was related, closed.

\begin{description}

\item \textbf{Parameters:} Human's phone number, his mark-answer to QA
questions, date when his sms message with answers arrived, time of this sms
message
\item \textbf{Precondition:}
\begin{itemize}
  \item the system is supposed to be created and initialized
  \item there exists human entry in system with given phone number
  \item the date and time the answer is declared is supposed to be in the past
  with respect to the current time known by the system
  \item the difference between the timestamp when QA survey was sent to human  
  (saved in ``ClosedTimestamp'' field of crisis in the system) and declared
  time of answer is less or equals to 5 hours (the system waits answers to each
  QA survey for 5 hours since it is sent)
\end{itemize}
\item \textbf{Post-condition:} there exist answers in the system to QA survey
question associated with crisis to which the human with given phone number was previously
associated
\item \textbf{Output messages:} If all field are filled in a valid way, the
message ``Answer sent!''. Otherwise - ``Incorrect data'' message is shown.

\item \textbf{Triggering:}
\begin{enumerate}
\item From within the crisis management Communication Company window press
``Send QA Survey answer'' button  
\item In pop-up window fill out all fields related to QA survey answers such as
phone number, answers message, date and time
\item Press ``Send answer'' button
\end{enumerate}

\end{description}

\section{Coordinator}

\subsection{GetAlertsSet}

Coordinator requests a detailed list of all the alerts having a specific status.

\begin{description}

\item \textbf{Parameters:} Alert Status (either pending, valid or invalid)
\item \textbf{Precondition:} The system is started and the coordinator logged
previously and did not log out.
\item \textbf{Post-condition:} Information about alerts with specified status
which exist in the system is sent to coordinator.
\item \textbf{Output messages:} Coordinator notified about each alert whose
information he obtained.

\item \textbf{Triggering:}
\begin{enumerate}
\item From within the crisis management Coordinator window choose ``alerts'' tab
\item Choose alert's type in top-down list.
\end{enumerate}

\end{description}

\subsection{GetCrisisSet}

Coordinator requests a detailed list of all the crises having a specific status.

\begin{description}

\item \textbf{Parameters:} Crisis Status (either pending, handled, solved or
closed)
\item \textbf{Precondition:} The system is started and the coordinator logged
previously and did not log out. 
\item \textbf{Post-condition:} Information about crises with specified status
which exist in the system is sent to coordinator.
\item \textbf{Output messages:} Coordinator notified about each crisis whose
information he obtained.

\item \textbf{Triggering:}
\begin{enumerate}
\item From within the crisis management Coordinator window choose ``crises'' tab
\item Choose crisis' type in top-down list.
\end{enumerate}

\end{description}

\subsection{ValidateAlert}

Coordinator marks specific alert as valid (is not a fake).

\begin{description}

\item \textbf{Parameters:} Alert's ID
\item \textbf{Precondition:} The system is started, the coordinator logged
previously and did not log out, and there exists alert with specified ID
\item \textbf{Post-condition:} alert with specified ID is marked as a valid
\item \textbf{Output messages:} the coordinator is notified about the
satisfaction of his request

\item \textbf{Triggering:}
\begin{enumerate}
\item From within the crisis management Coordinator window choose ``alerts'' tab
\item Choose alert's type 'pending' in top-down list
\item Choose alert which coordinator wants to validate
\item Press ``Validate'' button
\end{enumerate}

\end{description}

\subsection{InvalidateAlert}

Coordinator marks specific alert as invalid (is a fake).

\begin{description}

\item \textbf{Parameters:} Alert's ID
\item \textbf{Precondition:} The system is started, the coordinator logged
previously and did not log out, and there exists alert with specified ID
\item \textbf{Post-condition:} alert with specified ID is marked as a invalid
\item \textbf{Output messages:} the coordinator is notified about the
satisfaction of his request

\item \textbf{Triggering:}
\begin{enumerate}
\item From within the crisis management Coordinator window choose ``alerts'' tab
\item Choose alert's type 'pending' in top-down list
\item Choose alert which coordinator wants to invalidate
\item Press ``Invalidate'' button
\end{enumerate}

\end{description}

\subsection{SetCrisisHandler}

Coordinator declares himself as been the a handler of a crisis having the
specified id.

\begin{description}

\item \textbf{Parameters:} Crisis ID
\item \textbf{Precondition:} The system is started, the coordinator logged
previously and did not log out, and there exists crisis with specified ID
\item \textbf{Post-condition:} 
\begin{itemize}
  \item the crisis with specified ID is in 'handled' status and is associated to
  the Coordinator
  \item all the alerts related to this crisis are sent to the Coordinator such
  that he can decide how to handle them
  \item if the crisis was already handled by other coordianted then the
  associated actor is notified about the change of handler for one of his crisis
  (n.b. it might be the same even if not relevant) by message ``One of
  the crisis you were handling is now handled by one of your colleagues!''
  \item a message is sent to the communication company for any human related to
  an alert associated to the crisis. A human will receive as many messages as
  alerts he sent despite the fact that they might relate to the same crisis
  (i.e. one alert, one acknoledgement).
\end{itemize}
\item \textbf{Output messages:} ``You are now considered as handling the
crisis!'' message is shown to coordinator, and ``The handling of your alert by
our services is in progress!'' message is sent to Communication Company for any
human related to an alert associated to the crisis with specified ID

\item \textbf{Triggering:}
\begin{enumerate}
\item From within the crisis management Coordinator window choose ``crisis'' tab
\item Choose crisis's type in top-down list
\item Choose crisis which coordinator wants to handle
\item Press ``Handle crisis'' button
\end{enumerate}

\end{description}

\subsection{ReportOnCrisis}

Coordinator updates the textual information available for a specific crisis.

\begin{description}

\item \textbf{Parameters:} Crisis ID, New Crisis' comment
\item \textbf{Precondition:} The system is started, the coordinator logged
previously and did not log out, and there exists crisis with specified ID
\item \textbf{Post-condition:} the comment attribute of the crisis having the
given ID is replaced by the given one 
\item \textbf{Output messages:} the coordinator is automatically notified about
succesful update

\item \textbf{Triggering:}
\begin{enumerate}
\item From within the crisis management Coordinator window choose ``crisis'' tab
\item Choose crisis's type in top-down list
\item Choose crisis which coordinator wants to report
\item Press ``Report crisis'' button
\item In pop-up window fill the text area related to new crisis' comment
\item Press ``Report'' button to update crisis' comment 
\end{enumerate}

\end{description}

\subsection{SetCrisisStatus}

Coordinator defines the handling status of a specific crisis.

\begin{description}

\item \textbf{Parameters:} crisis ID, new crisis' status
\item \textbf{Precondition:} The system is started, the coordinator logged
previously and did not log out, and there exists crisis with specified ID
\item \textbf{Post-condition:} the crisis with specified ID is in specified
status and is associated to the Coordinator
\item \textbf{Output messages:} ``The crisis status has been updated!'' message
is shown to the coordinator

\item \textbf{Triggering:}
\begin{enumerate}
\item From within the crisis management Coordinator window choose ``crisis'' tab
\item Choose crisis's type in top-down list
\item Choose crisis which coordinator wants to report
\item Click ``Change crisis status'' button
\item In pop-up window change crisis' status to one which coordinator prefer
using top-down list
\item Press ``Change status'' button
\end{enumerate}

\end{description}

\subsection{CloseCrisis}

Coordinator marks specific crisis as closed. 

\begin{description}

\item \textbf{Parameters:} crisis ID
\item \textbf{Precondition:} The system is started, the coordinator logged
previously and did not log out, and there exists crisis with specified ID
\item \textbf{Post-condition:} 
\begin{itemize}
  \item the status of crisis with specified ID is ``closed''
  \item There is no coordinator declared in the system as associated to the
  crisis
  \item all the alert instances associated to this crisis do not belong any more
  to the system
  \item QA survey opened for 5 hours and QA survey sets of satisfaction single
  choice text questions are sent to all people associated with this crisis'
  alerts through corresponding communication company
\end{itemize}
\item \textbf{Output messages:} message ``The crisis is now closed!'' is shown
to the coordinator

\item \textbf{Triggering:}
\begin{enumerate}
\item From within the crisis management Coordinator window choose ``crisis'' tab
\item Choose crisis's type in top-down list
\item Choose crisis which coordinator wants to close
\item Press ``Close crisis'' button
\end{enumerate}

\end{description}

\section{Creator}

\subsection{CreateSystemAndEnvironment}

Creator initializes the system and the environment actors instances.

\begin{description}

\item \textbf{Parameters:} quantity of communication companies to create in the environment
\item \textbf{Precondition:} none
\item \textbf{Post-condition:}
\begin{itemize}
  \item system is initialized with the integer 1 for the crisis
  and alert counters used for their identifications, a value for the clock
  corresponding to a default inital time (i.e. January 1st, 1970), the crisis
  reminder period is set to 300 seconds, the maximum crisis reminder period is
  fixed to 1200 seconds (i.e. 20 minutes), an initial value for the automatic
  reminder period equal to the current date and time and the system is
  considered in a started state
  \item the environment for communication company actors, is
  made of instances allowing for receiving and sending messages to humans
  \item the environment for administrator actors is made of one instance
  \item the environment for activator actors is made of one instance allowing
  for automatic message sending based on current system’s and environment's
  state
\end{itemize}
\item \textbf{Output messages:} none

\item \textbf{Triggering:}
\begin{enumerate}
\item Creator runs system's server and desktop client
\end{enumerate}

\end{description}