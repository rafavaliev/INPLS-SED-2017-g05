\chapter{Introduction}
\label{chap:introduction}

\section{Scope}
This section has to provide the scope of the user's manual document.
In the following some opening statements to use when providing the
information corresponding to this section.

This document provides minimum acceptable information for knowing how to use the software system \mysystemname.
%Example: This document provides minimum acceptable information for knowing how
% to use the software system \mysystemname.


This document does not provide all details regarding \mysystemname.
 
This document is not intended to provide information about how to
 connect, deploy, configure, or use any external device or
 third-party software system that is required for the correct functioning of
 \mysystemname.
%Example: This document is not intended to provide information about how to
% connect, deploy, configure, or use any external device or
% third-party software system that is rqeuired for the correct funcitoning of
% \mysystemname.

 
This document may be used with other documents provided by third-party
companies to have an overall view and correct understanding of the environment
and procedures where the software system \mysystemname is aimed to be deployed and run.
%This document may be used with other documents provided by third-party
% companies to have an overall view and correct understanding of the environment
% and procedures where the software system \mysystemname is aimed to be deployed
% and run.




\section{Purpose}
In this section you explain the purpose (i.e. aim, objectives) of the user's
manual. In the following some examples of opening statements to be used in this
section.

The purpose of this document is to give some minimum amount of information
required for using the system.

This document defines such models as \gls{Environment
Model}, \gls{Concept Model}, etc. It also defines several views such as
\gls{Deployment View}, \gls{Implementation View}, etc.



\section{Intended audience}
Description of the categories of persons targeted by this document together with the description of how they are expected to exploit the content of the document.


\section{\mysystemname}
Brief overview of the software application domain and main purpose.


\subsection{Actors \& Functionalities}

% Overview of all the \textbf{\emph{\glspl{actor}}} interacting with the software
% being them either humans (called end-users in the standard
% \cite{IEEE-2001-userdocumentation}) or not. For each actor, describe the main
% software functions that are offered to him. Structure of this sub-section MUST
% be by actor/functionalities.

\subsubsection{Communication Company}
A company that has the capacity to ensure communication of information between
its customers and the \mysystemname system. 

\textbf{Functionalities}:

\begin{itemize}
  \item deliver SMS about possible crisis got from witness or victim to the
  \mysystemname's phone number in form of alert
  \item be notified when handling alert by Coordinator of \mysystemname system,
  sent by the Communication Company previously, is in progress
  \item transmit SMS messages from company that owns \mysystemname system to any
  human having an SMS compatible device accessible using a phone number
\end{itemize}

\subsubsection{Human}
Any person who considers himself related to a car crash either as a
witness, a victim or an anonymous person. 

\textbf{Functionalities}:

\begin{itemize}
  \item inform the \mysystemname system about the crisis situation he detected
  by sending SMS with information about the crisis to some Communication Company
  which will send an alert to \mysystemname system
  \item be notified that the ABC company has been informed about the situation
  \item be informed about the situation of the crisis he/she has related to as a
  victim or witness
  \item get QA survey (set of satisfaction single choice text questions with
  possible mark answer for each from 0 to 5) by SMS message when a crisis,
  associated with this Human, processed by \mysystemname's Coordinator. QA
  survey will be open (\mysystemname system will wait answer from Human) for 5
  hours.
  \item answer obtained QA survey by sending reply in form of corresponding
  mark-answers for each question separated by spaces
\end{itemize}

\subsubsection{Coordinator}
An employee of the company, owning \mysystemname system, being responsible of
handling one or several crisis.

\textbf{Functionalities}:

\begin{itemize}
  \item authenticate in \mysystemname system by providing valid (present in the
  system) login/password and, if login/password pair is valid, pressing on
  fingerprint scanner with his finger in special mobile app in the following 60
  seconds
  \item observe alerts by their status (pending/valid/invalid)
  \item observe crises by their status (pending/handled/solved/closed)
  \item validate/invalidate pending alert (mark alert as real or not)
  \item handle crisis
  \item leave free text comment on crisis
  \item change crisis's status
  \item close crisis (make crisis's status equal to 'closed')
\end{itemize}

\subsubsection{Administrator}
An employee of the company, owning \mysystemname system, being responsible of
administrating the system. 

\textbf{Functionalities}:

\begin{itemize}
  \item authenticate in \mysystemname system by providing valid (present in the
  system) login/password and, if login/password pair is valid, pressing on
  fingerprint scanner with his finger in special mobile app in the following 60
  seconds
  \item add or delete coordinators from the system and its environment
  \item be notified about intrusions (when some person tried to authenticate as
  an Coordinator, but provided invalid (not stored in \mysystemname system)
  login/password pair or his fingerprint scan didn't match one associated with
  corresponding (defined by login/password pair) coordinator)
  \item visualize satisfaction diagrams in form of eye chart showing relative
  amounts of each type of mark of for given QA survey question. The
  administrator can see such chart of question for all crises or for only
  specified one.
\end{itemize}

\subsubsection{Creator}
A technician who is installing the \mysystemname system on the targeted
deployment infrastructure.

\textbf{Functionalities}:

\begin{itemize}
  \item install the \mysystemname system
  \item define the values for the initial system’s state
  \item define the values for the initial system’s environment
\end{itemize}

\subsubsection{Activator}
A logical representation of the active part the \mysystemname system.
It represents an implicit stakeholder belonging to the system’s environment
that interacts with the iCrash system autonomously without the need of a
external entity. It is usually used for representing time triggered functionalities.

\textbf{Functionalities}:

\begin{itemize}
  \item communicate the current time to the system
  \item notify the administrator that some crisis are still pending for a too
  long time
\end{itemize}

\subsection{Operating environment}
Brief overview of the infrastructure on which the software is deployed and used.

\section{Document structure}  
Information on how this document is organised and it is expected to be
used. Recommendations on which members of the audience
should consult which sections of the document, and explanations about the used
notation (i.e. description of formats and conventions) must also be provided.





