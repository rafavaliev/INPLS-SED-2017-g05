\chapter{Introduction}
\label{chap:introduction}

\section{Scope}
This document provides all the information, which is required for the successful
usage of the \mysystemname software system.

This document does not contain information about any legal issues, which might
arise during operating of the \mysystemname software system, as well as
information regarding system deployment and configuration.
 
This document may be used with user manuals and specification documents provided
by third-party manufacturers, considered useful for the understanding of all the
infrastructure, required for \mysystemname functioning.

\section{Purpose}
The purpose of this document is to assist in use of \mysystemname software
system for all the persons intended (see ``Intended audience`` section) to use
this system.

This document defines all the ways users of the systems can interact with the
system by describing all possible software operations and explaining error
messages.

\section{Intended audience}
This document is supposed to be read by two categories of \mysystemname users,
both of these are employees of the ABC company:
\begin{itemize}
  \item Administrator - a person, who is responsible for adding and deleting of
  coordinators.
  \item Coordinators -  persons, who are responsible for crises handling and
  monitoring of existing alerts.
\end{itemize}


\section{\mysystemname}
The \mysystemname software system belongs to the Crisis Managment Systems
Domain. It is aimed to provide a convenient  way for crisis witnesses  and
victims to report about the crisis (with a possibility to do so anonymously),
and then to securely handle the crisis for a coordinator.

\subsection{Actors \& Functionalities}

% Overview of all the \textbf{\emph{\glspl{actor}}} interacting with the software
% being them either humans (called end-users in the standard
% \cite{IEEE-2001-userdocumentation}) or not. For each actor, describe the main
% software functions that are offered to him. Structure of this sub-section MUST
% be by actor/functionalities.

\subsubsection{Communication Company}
A company that has the capacity to ensure communication of information between
its customers and the \mysystemname system. 

\textbf{Functionalities}:

\begin{itemize}
  \item deliver SMS about possible crisis got from witness or victim to the
  \mysystemname's phone number in form of alert
  \item be notified when handling alert by Coordinator of \mysystemname system,
  sent by the Communication Company previously, is in progress
  \item transmit SMS messages from company that owns \mysystemname system to any
  human having an SMS compatible device accessible using a phone number
\end{itemize}

\subsubsection{Human}
Any person who considers himself related to a car crash either as a
witness, a victim or an anonymous person. 

\textbf{Functionalities}:

\begin{itemize}
  \item inform the \mysystemname system about the crisis situation he detected
  by sending SMS with information about the crisis to some Communication Company
  which will send an alert to \mysystemname system
  \item be notified that the ABC company has been informed about the situation
  \item be informed about the situation of the crisis he/she has related to as a
  victim or witness
  \item get QA survey (set of satisfaction single choice text questions with
  possible mark answer for each from 0 to 5) by SMS message when a crisis,
  associated with this Human, processed by \mysystemname's Coordinator. QA
  survey will be open (\mysystemname system will wait answer from Human) for 5
  hours.
  \item answer obtained QA survey by sending reply in form of corresponding
  mark-answers for each question separated by spaces
\end{itemize}

\subsubsection{Coordinator}
An employee of the company, owning \mysystemname system, being responsible of
handling one or several crisis.

\textbf{Functionalities}:

\begin{itemize}
  \item authenticate in \mysystemname system by providing valid (present in the
  system) login/password and, if login/password pair is valid, pressing on
  fingerprint scanner with his finger in special mobile app in the following 60
  seconds
  \item observe alerts by their status (pending/valid/invalid)
  \item observe crises by their status (pending/handled/solved/closed)
  \item validate/invalidate pending alert (mark alert as real or not)
  \item handle crisis
  \item leave free text comment on crisis
  \item change crisis's status
  \item close crisis (make crisis's status equal to 'closed')
\end{itemize}

\subsubsection{Administrator}
An employee of the company, owning \mysystemname system, being responsible of
administrating the system. 

\textbf{Functionalities}:

\begin{itemize}
  \item authenticate in \mysystemname system by providing valid (present in the
  system) login/password and, if login/password pair is valid, pressing on
  fingerprint scanner with his finger in special mobile app in the following 60
  seconds
  \item add or delete coordinators from the system and its environment
  \item be notified about intrusions (when some person tried to authenticate as
  an Coordinator, but provided invalid (not stored in \mysystemname system)
  login/password pair or his fingerprint scan didn't match one associated with
  corresponding (defined by login/password pair) coordinator)
  \item visualize satisfaction diagrams in form of eye chart showing relative
  amounts of each type of mark of for given QA survey question. The
  administrator can see such chart of question for all crises or for only
  specified one.
\end{itemize}

\subsubsection{Creator}
A technician who is installing the \mysystemname system on the targeted
deployment infrastructure.

\textbf{Functionalities}:

\begin{itemize}
  \item install the \mysystemname system
  \item define the values for the initial system’s state
  \item define the values for the initial system’s environment
\end{itemize}

\subsubsection{Activator}
A logical representation of the active part the \mysystemname system.
It represents an implicit stakeholder belonging to the system’s environment
that interacts with the iCrash system autonomously without the need of a
external entity. It is usually used for representing time triggered functionalities.

\textbf{Functionalities}:

\begin{itemize}
  \item communicate the current time to the system
  \item notify the administrator that some crisis are still pending for a too
  long time
\end{itemize}

\subsection{Operating environment}
This section describes the infrastructure required for deploying and running
client of the \mysystemname.\\

Processor : 2.66 GHz Intel  Core i5 or similar (minimum), 2.7 GHz Intel  Core i7
or similar (recommended).\\
Memory : 6GB (minimum), 8GB (recommended).\\
Disk space available: 30GB (minimum), 50GB (recommended).\\
Supported platforms: Windows 7,8. Mac OS X Yosemite. Linux Ubuntu (14.04 LTS).\\
Screen resolution: 1280x800 (minimum).\\
Third-party software: JRE 8.*, Cygwin's SSH (for Windows platform only).

\section{Document structure}  
The user manual consists of six sections: Information, Introduction, User guide,
Software operations, Error messages, and Appendix. All the sections of the
document should be read carefully by all the software users (both
administrator and coordinators).\\

Information section includes legal information (trademark, copyright etc.) and
contact information.\\

Introduction sections includes information about document's scope, purpose,
intended audience, \mysystemname domain information, actors and their software
functions description, infrastructure overview, and document structure.\\

User guide section includes description of general software use after it's
deployment and configuration.\\

Software operations section includes detailed explanation of \mysystemname
software operations (including a description of how to recognize if an operation
was completed successfully or abnormally terminated).\\

Error messages section includes detailed description of all known problems which
might arise during \mysystemname usage.\\

Appendix section includes some auxilary information.




